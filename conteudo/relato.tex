\chapter{Relato}

Durante todo o projeto da disciplina, nosso grupo teve a oportunidade de experimentar na prática o que foi abordado dentro de sala de aula, evidenciando o papel essencial da área de engenharia de requisitos dentro da engenharia de \textit{software}. Ao decorrer das aulas teóricas vimos que a engenharia de requisitos fez se necessária a partir do ponto que problemas iam surgindo no desenvolvimento de \textit{software} de larga escala, que exijiam uma lógica maior por trás do próprio produto final de \textit{software} em si, gerando dúvidas como "o que deve ser feito?" e "como deve ser feito?" para solucionar um problemas dentro de um projeto. A engenharia de requisitos surgiu deste contexto, de solucionar o problema de maneira mais correta possível e , consequentemente, gerar um \textit{software} de qualidade, pois a engenharia de requisitos está diretamente relacionada à busca de objetivos a serem atingidos pelo produto de \textit{software} a ser desenvolvido, ou seja, serve como um ponte que liga o problema do cliente à então solução de \textit{software}.

Dentro da engenharia de \textit{software} temos abordagens que se diferem umas das outras de acordo com as suas filosofias praticadas, porém ambas possuem um mesmo objetivo, chegar a uma solução de qualidade para um problema, neste caso as atividades da engenharia de requisitos são as mesmas em todas as abordagens da engenharia de \textit{software}, mudando apenas a maneira que uma ou outra atividade é realizada devido a essa variação de filosofias, e isso tudo ficou evidenciado para o nosso grupo, não apenas pelo fato de estarmos praticando um das abordagens, mas podendo ver o trabalho de outros grupos da disciplina, e os diferentes pontos de vista das abordagens sendo praticadas pelos mesmos.

Nosso projeto foi dividido em duas etapas práticas, na primeira etapa do projeto além de escolhermos a abordagem de acordo com o contexto do cliente que nos foi dado, levantamos um planejamento de como fariamos a segunda parte do projeto,definimos quais técnicas para a elicitação dos requisitos e as estratégias de rastreabilidade a serem seguidas no nosso trabalho, além da ferramenta para suporte de documentação de requisitos. Tudo feito com o objetivo de organizarmos e nos alhinharmos em relação ao o que será feito e nos prepararmos para como será feita a prática na segunda etapa. A parte prática do nosso projeto tem como intuito recriar um ambiente de trabalho profissional, onde temos pessoas diferentes com o mesmo objetivo, de chegar a uma boa solução de \textit{software} que resolva o problema do cliente.

Neste contexto realizamos reuniões para obter um entendimento comum entre os membros do grupo à respeito desse contexto do cliente, e tendo uma visão comum acerca do problema a ser resolvido, começamos a compreender as necessidades dos envolvidos e extrair as caracteristicas destas necessidades, para que os principais requisitos do \textit{software} a ser construído fossem elicitados podendo assim realizar o planejamento da primeira etapa do projeto.

Ao longo do projeto, também utilizamos as estratégias de rastreabilidade propostas na primeira etapa e conjunto com a ferramenta selecionada na mesma, entretanto a ferramenta não possuia o suporte necessário para manter a rastreabilidade proposta no planejamento do projeto, porém forneceu um grande auxílio para a documentação e manutenção dos requisitos do nosso projeto. A seleção dessa estratégia de rastreabilidade foi o suficiente para que mantivéssemos sempre uma visão clara da origem de um requisito ,característica e necessidade e queis delas estavam associadas.

No decorrer do projeto, nosso grupo procurou sempre manter o alinhamento de todos os membros juntos com os clientes e apesar de terem ocorrido alguns contratempos, sempre mantivemos o foco no objetivo a ser alcançado na disciplina, a experiência obtida durante a o processo do projeto da disciplina, que podemos considerar muito maior que a nota final do nosso trabalho.