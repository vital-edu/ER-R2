\chapter{Necessidades do Cliente}
\label{sec:nc}

Após o levantamento do problema central do cliente (Falta de controle do processo de desenvolvimento), utilizamos, de maneira predominante, a técnica de elicitação de brainstorm para levantamento das necessidades tentando manter coerência e coesão com o fishbone gerado na etapa de análise de problema.

\section{NC1 - Permitir o gerenciamento de funcionários}

É uma necessidade do cliente que o software gerencie a criação, desativação, alteração e consulta de funcionários. Esses funcionários devem ter papéis diferentes e permissões condizantes com seus papéis. 
\section{NC2 -Visualizar as habilidades técnicas e os horários disponíveis de cada desenvolvedor}

Deve ser possível o Acesso às informações de habilidades e horários, tendo em vista as permissões dos usuários (usuários sem permissão não podem ter acesso ou alterar certas informações).

\section{NC3 -Permitir o gerenciamento de múltiplos projetos}

Essa necessidade trata da criação e gerenciamento de projetos, alocação de membros para projetos, e definição de papéis internos.

\section{NC4 -Gerenciar tarefas de desenvolvimento de cada projeto}

Por fim o software deve oferecer a criação e gerenciamento de status de tarefas dentro de um projeto, responsaveis por tarefas, e progressos.
