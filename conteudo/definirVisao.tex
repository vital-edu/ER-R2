\section{Definição inicial do Documento de Visão}
\label{sec:visao}

Essa atividade usa os levantamentos de atores e requisitos das atividades anteriores do processo para a confecção inicial do documento de Visão (apendice \ref{sec:art-visao}). Dessa forma, esse documento mostra em um nível mais alto os requisitos e escopo do projeto.

Esse é um documento que estabelece um acordo sobre o problema a ser resolvido juntamente com as necessidades do cliente, além de levantar os envolvidos no contexto e quais serão os atores do sistema. Com relação a esse acordo, esse documento mostra as principais diferenças e semelhanças dos concorrentes no mercado e a perspectiva do produto.

\subsection{Problema e Necessidades}
A parte de problema e necessidades foram elicitadas como descrito nas seções \ref{sec:prob} e \ref{sec:nc}, para então serem transferidas para o documento de Visão.

\subsection{Contexto e Atores}
Tanto a parte de contexto do cliente, como ambiente do cliente e atores foram elicitadas como descrito na seção \ref{sec:elicitacao}, e então documentadas no documento de Visão

\subsection{Acordo com o Cliente}
Para entrar em acordo com o cliente, é necessário convencê-lo a não recorrer a concorrência e mostrar que o produto atende a suas expectativas. Dessa forma, no documento de Visão são mostradas vantagens e desvantagens de softwares semelhantes ao oferecido para o cliente, a fim de convencer a permanecer com a proposta. Além disso, é mostrada a perspectiva de atuação do produto, com o mesmo fim de convencer o cliente.
