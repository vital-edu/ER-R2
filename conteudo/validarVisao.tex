\chapter{Validar Visão}

Dado que o documento de visão apresenta o entendimento sobre as funcionalidades que o software deve possuir, podendo servir como um contrato do que deverá ser entregue, é fundamental que o mesmo seja validado com o cliente.Logo, após a elaboração do visão, o mesmo entrou em processo de validação com o cliente, que identificou algumas inconsistências, como a descrição dos papéis de alguns atores, necessidades do cliente, perspestiva do produto e alguns requisitos funcionais. 

Sendo assim, com o feedback da validação, foi feito um refinamento no documento de visão, refinando os itens citados anteriormente, para uma nova validação, onde o cliente então, aprovou o refinamento do documento e o processo seguiu adiante.
