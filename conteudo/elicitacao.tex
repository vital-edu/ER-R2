\section{Resultado das Técnicas de Elicitação}

Como descrito no trabalho 1, para elicitar adequadamente os requisitos de um software, é necessário a utilização de algumas técnicas de elicitação. Logo, neste capítulo serão descritos os resultados obtidos com as técnicas definidas no trabalho 1.

\subsection{Brainstorming}

Durante muitas reuniões entre a equipe de Requisitos e de MPR, aconteciam diversas sessões de brainstorming, onde tivemos uma visão melhor dos problemas atuais enfrentados dentro do nosso contexto(InSoft), e a partir deste momento, foram surgindo ideias a respeito de qual o problema que deveria ser atacado, o que poderia ou não ser resolvido com um software, possíveis soluções, soluções já existentes, entre outras coisas. A equipe em geral, não teve nenhum problema ao utilizar essa técnica, todos conseguiram falar e expressar suas respectivas ideias e opiniões.

Dessa forma, concluímos, que essa tećnica foi de extrema importância no decorrer do projeto, e trouxe resultado satisfatórios principalmente para o contexto inicial do projeto, onde as ideias e os problemas ainda eram muito incertos.

\subsection{Prototipação}

A partir da prototipação, foi possível descer um pouco o nível de abstração do sistema, onde podemos já pensar nas telas, e como as mesmas iam se comunicar, construindo uma primeira visão do sistema. A partir daí foi possível observar os pontos fortes e fracos do sistema, assim como um estudo detalhado dos requisitos, afim de revelar alguma inconsistência ou omissão.
Logo. essa técnica foi de extrema importância justamente pelo fato de possibilitar essa primeira visão do sistema, pois assim alguns erros e inconsistências foram atacados, antes mesmo da fase de implementação.

\subsection{Análise Competitiva}

Essa técnica foi importante, pois foi possível fazer uma análise com as soluções já existentes para o problema encontrado. Essa análise nos permitiu verificar pontos fortes e fracos das soluções, e quais necessidades essas soluções eram capazez de atender, e num geral, as soluções atendiam as necessidade de gerenciamento de projeto, como criação de projetos, criação de tarefas. Mas em contra partida, algumas necessidades como a visibilidade de horários disponíves, e de habilidades de cada membro, não foi atendida em nenhuma solução existente, e esses são pontos chaves que contribuíam diretamente para o problema maior que é a falta de controle do processo de desenvolvimento.

Dessa forma, concluímos, que essa técnica foi importante justamente para visualizarmos certas funcionalidades que existem em outras ferramentas, e que podemos implementar na nossa solução, e também o contrário, ou seja, funcionalidades que não existem nas ferramentas atuais, e que agregam valor para a solução construída.

Para visualizar a análise feita com as soluções existentes, acesse: \ref{subsec:Concorrentes}
