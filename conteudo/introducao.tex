\chapter{Introdução}

A engenharia de \emph{software} tem-se mostrado fundamental em um mundo cada vez mais globalizado e ágil. A necessidade de se investir na melhoria e informatização de processos de negócio tem atraído a busca pelo desenvolvimento de sistemas de informação que consigam entregar \emph{software} de qualidade a preços acessíveis e prazos coerentes, sendo a engenharia de \emph{software} a área do conhecimento que procura satisfazer essas demandas.

Dentro da engenharia de \emph{software}, a disciplina de engenharia de requisítos trata das necessidades dos clientes que devem ser satisfeitas pelo sistema de \emph{software}, sendo reponsável por elicitar, analisar, negociar, documentar, gerenciar, verificar e validar os requisitos de \emph{software}, partindo das necessidades de negócio, dando origem aos requisitos  de usuário, requisitos de sistema, requisitos funcionais e não-funcionais.

O intuito deste trabalho é conseguir entender a importância da disciplina de engenharia de requisitos no desenvolvimento de \emph{softwares} de qualidade que consigam entregar valor de negócio ao usuário, e a partir deste entendimento, conseguir aplicar de forma eficaz os conhecimentos adquiridos em um projeto de desenvolvimento de \emph{software} em um contexto real de negócio.

Por fim, este relatório tenta demonstrar o processo de aprendizado prático obtido na execução do trabalho, e está organizado no seguinte formato:

\begin{description}
  \item [Adaptações e Mudanças:] Descrição e justificativa das mudanças que ocorreram no projeto após o planejamento inicial.
  \item [Planejamento do Trabalho:] Descrição do planejamento da execução do trabalho e visualização do cronograma de atividades.
  \item [Processo de Engenharia de Requisitos:] Descrição da aplicação prática de cada atividade definida no processo de engenharia de requisitos utilizada no projeto.
  \item [Experiência na Execução do Trabalho:] Relato sobre o desenvolvimento do trabalho e experiências adquiridas.
  \item [Engenharia de Requisitos - Relato:] Relato sobe as dificuldades encontradas na aplicação dos conceitos de engenharia de requisitos na produção de um sistema de informação e aprendizados adquiridos durante o processo.
\end{description}
