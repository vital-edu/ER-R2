\chapter{Priorização dos Requisitos}
\label{sec:priorizacao}

A atividade de priorização foi inicialmente elaborada a partir de reuniões com a equipe de MPR, em que foram levantas as necessidades que mais agregassem valor ao negócio da inSoft.

Com isso, chegou-se ao entendimento de que as necessidades que deveriam ser atendidas de forma prioritária seriam:

\begin{itemize}
  \item Gerenciar tarefas de desenvolvimento de cada projeto
  \item Visualizar as habilidades técnicas e os horários disponíveis de cada de desenvolvedor
\end{itemize}

De posse dessas informações, foi realizado uma reunião com o professor, em que o mesmo decidiu em conjunto com a parte da equipe de requisitos que os casos de uso que deveriam ser implementados, atendendo parcialmente às necessidades supracitadas, seriam:

\begin{itemize}
  \item Criar projeto
  \item Editar projeto
  \item Visualizar projeto
  \item Arquivar projeto
  \item Criar tarefa
  \item Editar tarefa
  \item Visualizar tarefa
  \item Arquivar tarefa
\end{itemize}

Durante a reunião foram feitas críticas positivas aos casos de usos descritos, e as devidas correções foram realizadas, estando os casos de uso priorizados, especificados no documento de visão, que se encontra no apêndice \ref{sec:esp-req}.
