\chapter{Relato de Experiência da Execução do Trabalho}

O processo de execução do trabalho de Requisitos de \emph{Software} tornou-se uma experiência desafiadora e satisfatória. A falta de tempo para os integrantes da equipe interagir, tornou-se o maior problema durante a execução do projeto. No entanto, deve-se mencionar que a equipe conseguiu de forma efetiva superar essa barreira, valendo-se do uso constante de reuniões \emph{online}.

Para evitar que as tarefas fossem desempenhadas de forma desproporcional e sem controle, foi definido desde o início que evitaría-se ao máximo que tarefas importantes fossem desempenhada individualmente. As reuniões contantes, mesmo que \emph{online}, garantiram que essa meta fosse alcançada.

No início da primeira etapa do projeto, em que ainda não se havia adotado essa estratégia, demoramos para iniciar as atividades e estabelecer um cronograma. Porém, após entendermos a importância de estabelecer metas e prazos, a utilização constante do cronograma, bem como atas e pautas para as reuniões, permitiu que conseguissemos desenvolver a primeira etapa do projeto de forma organizada e sob controle.

Embora houvesse atrasos na execução das atividades, ter ciência do atraso nos fez corrigir os erros, replanejar as atividades e conseguir cumprir com as metas a longo prazo. Sendo assim, o planejamento foi crucial, e talvez fator decisivo para o sucesso  na execução  da primeira etapa do trabalho.

A segunda etapa do projeto foi atrasada pela entrega do \emph{to be} por parte de MPR, o que provocou um impacto significativo no planejamento das atividades. Mas esse não foi o único problema que a equipe enfrentou na segunda etapa. A constante dissonância entre os membros de MPR e a falta de introsamento entre a equipe de requisitos e de MPR atrapalhou o desenvolvimento das atividades. Mas superada as advercidades, conseguimos prosseguir.

As reuniões contantes com o monitor de requisitos foi prepoderante para guiar-nos e para corrigir nossos erros, tanto de planejamento como de execução das atividades. Nunca hesitamos em pedir o esclarecimento ao monitor sobre as atividades que executávamos e temos que deixar registrado a nossa gratidão pela solicitude do monitor. Entendemos que sem ele não chegaríamos tão longe.

Por fim, é importante salientar que todo a execução do trabalho foi deveras cansativo e desgastante, mas que no final, a perseverança venceu, e assumimos a importante experiência ganha durante o processo.
