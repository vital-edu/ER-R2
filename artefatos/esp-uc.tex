\chapter{Especificação de Caso de Uso}


\section{UC01 -Criar Projeto}

O objetivo deste caso de uso é permitir ao diretor de desenvolvimento criar projetos dentro do sistema.

\subsection{Atores}

\begin{enumerate}
  \item Diretor de desenvolvimento
\end{enumerate}

\subsection{Pré-Condições}

\begin{enumerate}
  \item O diretor de desenvolvimento deverá estar previamente logado no sistema.
\end{enumerate}

\subsection{Fluxo Principal}

O caso de uso se inicia quando o diretor de desenvolvimento seleciona a opção  de criar um projeto. [RN 1]

\begin{enumerate}
  \item O diretor de desenvolvimento seleciona a opção de criar projeto.
  \item O sistema apresenta a tela de criação de projeto para o diretor de desenvolvimento.
  \item Dentro da tela de criação de projeto o diretor de desenvolvimento preenche as informações necessárias para criação do projeto.
  \item O sistema retorna ao diretor de desenvolvimento a informação de que o projeto foi criado com sucesso.
  \item O diretor de desenvolvimento é então encaminhado pelo sistema à tela dos projetos cadastrados.
  \item O caso de uso é encerrado.
\end{enumerate}

\subsection{Fluxos Alternativos}

\subsubsection{Cancelar criação de projeto}
Este fluxo se inicia quando o diretor de desenvolvimento aciona o botão “CANCELAR” dentro da tela de criação do projeto.

\begin{enumerate}
  \item O sistema retorna ao diretor de desenvolvimento uma caixa de diálogo perguntando se ele deseja fazer está ação de cancelamento ou não.
  \item O diretor de desenvolvimento seleciona a opção “SIM” dentro da caixa de diálogo.
  \item O diretor de desenvolvimento é então encaminhado pelo sistema à tela de seleção de projetos.
\end{enumerate}

\subsubsection{Informações não preenchidas}
Este fluxo se inicia quando no passo 3 do fluxo principal não foram informados todos os campos obrigatórios para criação do projeto.

\begin{enumerate}
  \item O sistema retorna ao diretor de desenvolvimento a informação de que houve um erro na autenticação relacionado aos dados informados.
  \item O fluxo retorna ao passo 2 do fluxo principal.
\end{enumerate}

\subsubsection{Projeto existente}
Este fluxo se inicia quando no passo 3 do fluxo principal foram inseridas informações idênticas as de outro projeto já existente no sistema. 

\begin{enumerate}
  \item O sistema retorna ao diretor de desenvolvimento a informação de que o projeto já existe no sistema.
  \item O fluxo retorna ao passo 5 do fluxo principal.
\end{enumerate}


\section{UC02 -Editar Projeto}

O objetivo deste caso de uso é permitir ao diretor de desenvolvimento editar os projetos dentro do sistema para atualizar informações sobre o mesmo.

\subsection{Atores}

\begin{enumerate}
  \item Diretor de desenvolvimento.
\end{enumerate}

\subsection{Pré-Condições}
\begin{enumerate}
  \item O diretor de desenvolvimento devera estar previamente logado no sistema.
  \item Deve existir um projeto cadastrado na base de dados do sistema.
\end{enumerate}

\subsection{Fluxo Principal}
O caso de uso se inicia quando o diretor de desenvolvimento se encontra na tela de visualização do projeto.

\begin{enumerate}
  \item O diretor de desenvolvimento seleciona a opção de editar projeto
  \item O sistema carrega a tela de edição do projeto com todas as informações do projeto.
  \item Dentro da tela de edição do projeto, o diretor de desenvolvimento edita os campos de informação que deseja alterar.
  \item O diretor de desenvolvimento aciona a opção de salvar as informações
  \item O sistema retorna ao diretor de desenvolvimento que as informações foram alteradas com sucesso.
  \item O diretor de desenvolvimento é encaminhado pelo sistema à tela de visualização do projeto.
  \item O caso de uso é encerrado.
\end{enumerate}

\subsection{Fluxos Alternativos}

\subsubsection{Cancelar a edição do projeto}
Este fluxo se inicia quando o diretor de desenvolvimento aciona a opção “cancelar” dentro da tela de edição do projeto.

\begin{enumerate}
  \item O sistema retorna ao diretor de desenvolvimento uma caixa de diálogo perguntando se ele deseja fazer está ação de cancelamento ou não.
  \item O diretor de desenvolvimento seleciona a opção “SIM” dentro da caixa de diálogo.
  \item O diretor de desenvolvimento é então encaminhado pelo sistema à tela de visualização de projeto.
\end{enumerate}

\subsubsection{Informações preenchidas incorretamente}
Este fluxo se inicia quando no passo 4 do fluxo principal quando o sistema identificar que campos não foram preenchidos corretamente.

\begin{enumerate}
  \item O sistema retorna ao diretor de desenvolvimento a informação de que houve um erro na validação das informações fornecidas.
  \item O sistema exibe todos os erros de validação encontrados.
  \item O fluxo retorna ao passo 3 do fluxo principal. 
\end{enumerate}

\section{UC03 - Visualizar Projeto}

Esse caso de uso descreve como qualquer usuário usa a aplicação para exibir as atividades do Projeto.

\subsection{Atores}
\begin{enumerate}
  \item Usuário 
\end{enumerate}
\subsection{Pré-Condições}

\begin{enumerate}
  \item O usuário deve estar logado.
  \item Deve existir um projeto cadastrado na base de dados da aplicação. 
\end{enumerate}

\subsection{Fluxo Principal}
O caso de uso começa quando o usuário acessa a tela de seleção de projetos.

\begin{enumerate}
  \item O usuário seleciona a opção de visualizar o quadro de tarefas de um projeto dentre os listados.
  \item O sistema retorna ao usuário o quadro de tarefas do projeto [RN1] [RN2].
  \item O caso de uso é encerrado. 
\end{enumerate}

\subsection{Fluxos Alternativos}

\subsubsection{Visualizar informações do projeto}
Este fluxo se inicia no passo 1 do fluxo principal quando o usuário ao invés de selecionar a opção de visualizar o quadro de tarefas de um projeto, seleciona a opção de visualizar as informações de um projeto

\begin{enumerate}
  \item O sistema retorna ao usuário as informações do sistema [RN3]
  \item O caso de uso é encerrado
\end{enumerate}

\section{UC04 - Arquivar Projeto}
Esse caso de Uso descreve como Diretor de Desenvolvimento usa a aplicação para arquivar um projeto existente.

\subsection{Atores}

\begin{enumerate}
  \item Diretor de Desenvolvimento
\end{enumerate}

\subsection{Pré-Condições}
\begin{enumerate}
  \item Deve existir um projeto cadastrado na base de dados da aplicação.
  \item Usuário logado como Diretor de Desenvolvimento.
\end{enumerate}

\subsection{Fluxo Principal}
O caso de uso começa quando o diretor de desenvolvimento acessa a tela de seleção de projetos.

\begin{enumerate}
  \item O diretor de desenvolvimento seleciona a opção de arquivar o projeto. [RN4]
  \item O sistema retorna uma mensagem perguntando ao diretor de desenvolvimento se ele deseja concluir a operação.
  \item O sistema retorna uma mensagem de confirmação informando que o arquivamento foi realizado com sucesso.
  \item O diretor de desenvolvimento é redirecionado a tela de seleção de projetos.
  \item O caso de uso é encerrado
\end{enumerate}

\subsection{Fluxos Alternativos}

\subsubsection{Desarquivar projeto}
O caso de uso começa quando o Diretor de Desenvolvimento acessa a tela de seleção de projetos.

\begin{enumerate}
  \item O usuário seleciona a opção visualizar projetos arquivados.
  \item O sistema redireciona o usuário para a tela de projetos arquivados [RN5]
  \item O usuário seleciona um projeto a ser desarquivado e clica em desarquivar.
  \item O sistema retorna a informação de que o projeto foi desarquivado com sucesso.
  \item O usuário é redirecionado para a tela de seleção de projetos.
  \item O projeto desarquivado é listado junto com os outros projetos na tela.
  \item O caso de uso é encerrado.
\end{enumerate}

\subsubsection{Erro na operação de arquivamento}
O caso de uso é iniciado após o passo 2 do fluxo principal quando o diretor de desenvolvimento confirma o arquivamento do projeto e por alguma razão a operação não é efetuada com sucesso

\begin{enumerate}
  \item O sistema retorna uma mensagem informando que não foi possível realizar a operação de arquivamento do projeto.
  \item O fluxo retorna ao passo 1 do fluxo principal 
\end{enumerate}

\subsubsection{Erro na operação de desarquivamento}
O caso de uso é iniciado após o passo 3 do fluxo alternativo “desarquivar projeto” quando o diretor de desenvolvimento confirma o desarquivamento do projeto e por alguma razão a operação não é efetuada com sucesso.

\begin{enumerate}
  \item O sistema retorna uma mensagem informando que não foi possível realizar a operação de desarquivamento do projeto.
  \item O fluxo retorna ao passo 3 do fluxo principal.
\end{enumerate}


\section{UC05 -Criar Tarefa}
O objetivo deste caso de uso é permitir ao desenvolvedor criar uma tarefa no sistema.

\subsection{Atores}

\begin{enumerate}
  \item Desenvolvedor
\end{enumerate}

\subsection{Pré-Condições}
\begin{enumerate}
  \item Existir um projeto já cadastrado na base de dados.
  \item Existir um usuário já cadastrado na base de dados.
\end{enumerate}

\subsection{Fluxo Principal}
O caso de uso se inicia quando um usuário clica para criar uma tarefa em algum projeto no sistema.

\begin{enumerate}
  \item O usuário visualiza todos os projetos do sistema.
  \item O usuário escolhe qual dos projetos o qual ele irá criar uma tarefa.
  \item O usuário aciona o botão “adicionar uma tarefa” na tela “Tarefas”.
  \item O usuário é direcionado para a página de criação da tarefa.
  \item O usuário preenche as informações necessárias sobre tal projeto
  \item O sistema adiciona tal tarefa no banco de dados relacionando com o projeto proposto.
  \item A tela deve ser redirecionada para os quadros do projeto com a atividade recém criada na parte “to do” (para fazer).
  \item O caso de uso é encerrado. 
\end{enumerate}

\subsection{Fluxos Alternativos}

\subsubsection{Cancelar Criação de Tarefa}

\begin{enumerate}
  \item O sistema retorna ao desenvolvedor uma caixa de diálogo perguntando se ele deseja fazer está ação de cancelamento ou não.
  \item O desenvolvedor seleciona a opção “SIM” dentro da caixa de diálogo.
  \item O desenvolvedor é então encaminhado pelo sistema à tela do projeto previamente escolhido.
  \item O caso de uso é encerrado.
\end{enumerate}

\subsubsection{Preencher novamente informações para revalidar}

\begin{enumerate}
  \item O sistema retorna ao desenvolvedor a informação de que houve um erro na autenticação relacionado aos dados informados.
  \item O fluxo retorna ao passo 4 do fluxo principal.
\end{enumerate}

\subsubsection{Tarefa Duplicada}

\begin{enumerate}
  \item Este fluxo se inicia quando no passo 5 do fluxo principal foram inseridas informações idênticas as de outra atividade (do mesmo projeto) já existente no sistema.
  \item O sistema retorna ao desenvolvedor a informação de que a tarefa já existe no sistema para tal projeto.
  \item O caso de uso é encerrado.
\end{enumerate}


\section{UC06 -}


\subsection{Atores}

\begin{enumerate}
  \item
\end{enumerate}

\subsection{Pré-Condições}
\begin{enumerate}
  \item
\end{enumerate}

\subsection{Fluxo Principal}
\begin{enumerate}
  \item
\end{enumerate}

\subsection{Fluxos Alternativos}

\subsubsection{--}

\begin{enumerate}
  \item
\end{enumerate}

\section{UC07 -}


\subsection{Atores}

\begin{enumerate}
  \item
\end{enumerate}

\subsection{Pré-Condições}
\begin{enumerate}
  \item
\end{enumerate}

\subsection{Fluxo Principal}
\begin{enumerate}
  \item
\end{enumerate}

\subsection{Fluxos Alternativos}

\subsubsection{--}

\begin{enumerate}
  \item
\end{enumerate}

\section{UC08 -}


\subsection{Atores}

\begin{enumerate}
  \item
\end{enumerate}

\subsection{Pré-Condições}
\begin{enumerate}
  \item
\end{enumerate}

\subsection{Fluxo Principal}
\begin{enumerate}
  \item
\end{enumerate}

\subsection{Fluxos Alternativos}

\subsubsection{--}

\begin{enumerate}
  \item
\end{enumerate}

