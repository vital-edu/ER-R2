\chapter{Especificação de Caso de Uso}
\label{sec:esp-req}

\section{UC01 -Criar Projeto}

O objetivo deste caso de uso é permitir ao diretor de desenvolvimento criar projetos dentro do sistema.

\subsection{Atores}

\begin{enumerate}
  \item Diretor de desenvolvimento
\end{enumerate}

\subsection{Pré-Condições}

\begin{enumerate}
  \item O diretor de desenvolvimento deverá estar previamente logado no sistema.
\end{enumerate}

\subsection{Fluxo Principal}

O caso de uso se inicia quando o diretor de desenvolvimento seleciona a opção  de criar um projeto. [RN 1]

\begin{enumerate}
  \item O diretor de desenvolvimento seleciona a opção de criar projeto.
  \item O sistema apresenta a tela de criação de projeto para o diretor de desenvolvimento.
  \item Dentro da tela de criação de projeto o diretor de desenvolvimento preenche as informações necessárias para criação do projeto.
  \item O sistema retorna ao diretor de desenvolvimento a informação de que o projeto foi criado com sucesso.
  \item O diretor de desenvolvimento é então encaminhado pelo sistema à tela dos projetos cadastrados.
  \item O caso de uso é encerrado.
\end{enumerate}

\subsection{Fluxos Alternativos}

\subsubsection{Cancelar criação de projeto}
Este fluxo se inicia quando o diretor de desenvolvimento aciona o botão “CANCELAR” dentro da tela de criação do projeto.

\begin{enumerate}
  \item O sistema retorna ao diretor de desenvolvimento uma caixa de diálogo perguntando se ele deseja fazer está ação de cancelamento ou não.
  \item O diretor de desenvolvimento seleciona a opção “SIM” dentro da caixa de diálogo.
  \item O diretor de desenvolvimento é então encaminhado pelo sistema à tela de seleção de projetos.
\end{enumerate}

\subsubsection{Informações não preenchidas}
Este fluxo se inicia quando no passo 3 do fluxo principal não foram informados todos os campos obrigatórios para criação do projeto.

\begin{enumerate}
  \item O sistema retorna ao diretor de desenvolvimento a informação de que houve um erro na autenticação relacionado aos dados informados.
  \item O fluxo retorna ao passo 2 do fluxo principal.
\end{enumerate}

\subsubsection{Projeto existente}
Este fluxo se inicia quando no passo 3 do fluxo principal foram inseridas informações idênticas as de outro projeto já existente no sistema.

\begin{enumerate}
  \item O sistema retorna ao diretor de desenvolvimento a informação de que o projeto já existe no sistema.
  \item O fluxo retorna ao passo 5 do fluxo principal.
\end{enumerate}


\section{UC02 -Editar Projeto}

O objetivo deste caso de uso é permitir ao diretor de desenvolvimento editar os projetos dentro do sistema para atualizar informações sobre o mesmo.

\subsection{Atores}

\begin{enumerate}
  \item Diretor de desenvolvimento.
\end{enumerate}

\subsection{Pré-Condições}
\begin{enumerate}
  \item O diretor de desenvolvimento devera estar previamente logado no sistema.
  \item Deve existir um projeto cadastrado na base de dados do sistema.
\end{enumerate}

\subsection{Fluxo Principal}
O caso de uso se inicia quando o diretor de desenvolvimento se encontra na tela de visualização do projeto.

\begin{enumerate}
  \item O diretor de desenvolvimento seleciona a opção de editar projeto
  \item O sistema carrega a tela de edição do projeto com todas as informações do projeto.
  \item Dentro da tela de edição do projeto, o diretor de desenvolvimento edita os campos de informação que deseja alterar.
  \item O diretor de desenvolvimento aciona a opção de salvar as informações
  \item O sistema retorna ao diretor de desenvolvimento que as informações foram alteradas com sucesso.
  \item O diretor de desenvolvimento é encaminhado pelo sistema à tela de visualização do projeto.
  \item O caso de uso é encerrado.
\end{enumerate}

\subsection{Fluxos Alternativos}

\subsubsection{Cancelar a edição do projeto}
Este fluxo se inicia quando o diretor de desenvolvimento aciona a opção “cancelar” dentro da tela de edição do projeto.

\begin{enumerate}
  \item O sistema retorna ao diretor de desenvolvimento uma caixa de diálogo perguntando se ele deseja fazer está ação de cancelamento ou não.
  \item O diretor de desenvolvimento seleciona a opção “SIM” dentro da caixa de diálogo.
  \item O diretor de desenvolvimento é então encaminhado pelo sistema à tela de visualização de projeto.
\end{enumerate}

\subsubsection{Informações preenchidas incorretamente}
Este fluxo se inicia quando no passo 4 do fluxo principal quando o sistema identificar que campos não foram preenchidos corretamente.

\begin{enumerate}
  \item O sistema retorna ao diretor de desenvolvimento a informação de que houve um erro na validação das informações fornecidas.
  \item O sistema exibe todos os erros de validação encontrados.
  \item O fluxo retorna ao passo 3 do fluxo principal.
\end{enumerate}

\section{UC03 - Visualizar Projeto}

Esse caso de uso descreve como qualquer usuário usa a aplicação para exibir as atividades do Projeto.

\subsection{Atores}
\begin{enumerate}
  \item Usuário
\end{enumerate}
\subsection{Pré-Condições}

\begin{enumerate}
  \item O usuário deve estar logado.
  \item Deve existir um projeto cadastrado na base de dados da aplicação.
\end{enumerate}

\subsection{Fluxo Principal}
O caso de uso começa quando o usuário acessa a tela de seleção de projetos.

\begin{enumerate}
  \item O usuário seleciona a opção de visualizar o quadro de tarefas de um projeto dentre os listados.
  \item O sistema retorna ao usuário o quadro de tarefas do projeto [RN1] [RN2].
  \item O caso de uso é encerrado.
\end{enumerate}

\subsection{Fluxos Alternativos}

\subsubsection{Visualizar informações do projeto}
Este fluxo se inicia no passo 1 do fluxo principal quando o usuário ao invés de selecionar a opção de visualizar o quadro de tarefas de um projeto, seleciona a opção de visualizar as informações de um projeto

\begin{enumerate}
  \item O sistema retorna ao usuário as informações do sistema [RN3]
  \item O caso de uso é encerrado
\end{enumerate}

\section{UC04 - Arquivar Projeto}
Esse caso de Uso descreve como Diretor de Desenvolvimento usa a aplicação para arquivar um projeto existente.

\subsection{Atores}

\begin{enumerate}
  \item Diretor de Desenvolvimento
\end{enumerate}

\subsection{Pré-Condições}
\begin{enumerate}
  \item Deve existir um projeto cadastrado na base de dados da aplicação.
  \item Usuário logado como Diretor de Desenvolvimento.
\end{enumerate}

\subsection{Fluxo Principal}
O caso de uso começa quando o diretor de desenvolvimento acessa a tela de seleção de projetos.

\begin{enumerate}
  \item O diretor de desenvolvimento seleciona a opção de arquivar o projeto. [RN4]
  \item O sistema retorna uma mensagem perguntando ao diretor de desenvolvimento se ele deseja concluir a operação.
  \item O sistema retorna uma mensagem de confirmação informando que o arquivamento foi realizado com sucesso.
  \item O diretor de desenvolvimento é redirecionado a tela de seleção de projetos.
  \item O caso de uso é encerrado
\end{enumerate}

\subsection{Fluxos Alternativos}

\subsubsection{Desarquivar projeto}
O caso de uso começa quando o Diretor de Desenvolvimento acessa a tela de seleção de projetos.

\begin{enumerate}
  \item O usuário seleciona a opção visualizar projetos arquivados.
  \item O sistema redireciona o usuário para a tela de projetos arquivados [RN5]
  \item O usuário seleciona um projeto a ser desarquivado e clica em desarquivar.
  \item O sistema retorna a informação de que o projeto foi desarquivado com sucesso.
  \item O usuário é redirecionado para a tela de seleção de projetos.
  \item O projeto desarquivado é listado junto com os outros projetos na tela.
  \item O caso de uso é encerrado.
\end{enumerate}

\section{UC05 -Criar Tarefa}
O objetivo deste caso de uso é permitir ao desenvolvedor criar uma tarefa no sistema.

\subsection{Atores}

\begin{enumerate}
  \item Desenvolvedor
\end{enumerate}

\subsection{Pré-Condições}
\begin{enumerate}
  \item Existir um projeto já cadastrado na base de dados.
  \item Existir um usuário já cadastrado na base de dados.
\end{enumerate}

\subsection{Fluxo Principal}
O caso de uso se inicia quando um usuário clica para criar uma tarefa em algum projeto no sistema.

\begin{enumerate}
  \item O usuário visualiza todos os projetos do sistema.
  \item O usuário escolhe qual dos projetos o qual ele irá criar uma tarefa.
  \item O usuário aciona o botão “adicionar uma tarefa” na tela “Tarefas”.
  \item O usuário é direcionado para a página de criação da tarefa.
  \item O usuário preenche as informações necessárias sobre tal projeto
  \item O sistema adiciona tal tarefa no banco de dados relacionando com o projeto proposto.
  \item A tela deve ser redirecionada para os quadros do projeto com a atividade recém criada na parte “to do” (para fazer).
  \item O caso de uso é encerrado.
\end{enumerate}

\subsection{Fluxos Alternativos}

\subsubsection{Cancelar Criação de Tarefa}

\begin{enumerate}
  \item O sistema retorna ao desenvolvedor uma caixa de diálogo perguntando se ele deseja fazer está ação de cancelamento ou não.
  \item O desenvolvedor seleciona a opção “SIM” dentro da caixa de diálogo.
  \item O desenvolvedor é então encaminhado pelo sistema à tela do projeto previamente escolhido.
  \item O caso de uso é encerrado.
\end{enumerate}

\subsubsection{Preencher novamente informações para revalidar}

\begin{enumerate}
  \item O sistema retorna ao desenvolvedor a informação de que houve um erro na autenticação relacionado aos dados informados.
  \item O fluxo retorna ao passo 4 do fluxo principal.
\end{enumerate}

\subsubsection{Tarefa Duplicada}

\begin{enumerate}
  \item Este fluxo se inicia quando no passo 5 do fluxo principal foram inseridas informações idênticas as de outra atividade (do mesmo projeto) já existente no sistema.
  \item O sistema retorna ao desenvolvedor a informação de que a tarefa já existe no sistema para tal projeto.
  \item O caso de uso é encerrado.
\end{enumerate}


\section{UC06 - Editar Tarefa}
O objetivo deste caso de uso é permitir ao desenvolvedor editar as tarefas dentro de um projeto de desenvolvimento para atualizar informações sobre a mesma.

\subsection{Atores}

\begin{enumerate}
  \item Usuário
\end{enumerate}

\subsection{Pré-Condições}
\begin{enumerate}
  \item O diretor de desenvolvimento devera estar previamente logado no sistema.
  \item Uma tarefa deve estar cadastrada na base de dados do sistema.
\end{enumerate}

\subsection{Fluxo Principal}
O caso de uso se inicia quando o diretor de desenvolvimento se encontra na tela de visualização da tarefa.

\begin{enumerate}
  \item O desenvolvedor seleciona a opção de editar a tarefa
  \item O sistema carrega a tela de edição da tarefa com todas as informações da tarefa.
  \item Dentro da tela de edição da tarefa, o desenvolvedor edita os campos de informação que deseja alterar
  \item O desenvolvedor aciona a opção de salvar as informações
  \item O sistema retorna ao desenvolvedor que as informações foram alteradas com sucesso.
  \item O diretor de desenvolvimento é encaminhado pelo sistema à tela de visualização da tarefa.
  \item O caso de uso é encerrado.
\end{enumerate}

\subsection{Fluxos Alternativos}

\subsubsection{Cancelar a edição da tarefa}
Este fluxo se inicia quando o desenvolvedor aciona a opção “cancelar” dentro da tela de edição da tarefa.

\begin{enumerate}
  \item O sistema retorna ao desenvolvedor uma caixa de diálogo perguntando se ele deseja fazer está ação de cancelamento ou não.
  \item O desenvolvedor seleciona a opção “SIM” dentro da caixa de diálogo
  \item O diretor de desenvolvimento é então encaminhado pelo sistema à tela de visualização da tarefa
\end{enumerate}

\subsubsection{Informações preenchidas incorretamente}
Este fluxo se inicia quando no passo 4 do fluxo principal quando o sistema identificar que campos não foram preenchidos corretamente.

\begin{enumerate}
  \item O sistema retorna ao ddesenvolvedor a informação de que houve um erro na validação das informações fornecidas
  \item O sistema exibe todos os erros de validação encontrados
  \item O fluxo retorna ao passo 3 do fluxo principal
\end{enumerate}


\section{UC07 -Visualizar Tarefa}
O objetivo deste caso de uso é permitir ao usuário visualizar as informações sobre determinada tarefa dentro de um projeto.

\subsection{Atores}

\begin{enumerate}
  \item Usuário
\end{enumerate}

\subsection{Pré-Condições}
\begin{enumerate}
  \item O usuário deve estar logado no sistema.
  \item Existir uma tarefa cadastrada dentro de um projeto na base de dados.
\end{enumerate}

\subsection{Fluxo Principal}
O caso de uso começa quando o usuário visualiza o quadro de tarefas de um projeto

\begin{enumerate}
  \item O usuário seleciona uma das tarefas existentes no quadro de tarefas.
  \item O sistema exibe as informações da tarefa (RN )
  \item O caso de uso é encerrado
\end{enumerate}

\subsection{Fluxos Alternativos}
Não se aplica.

\section{UC08 -Arquivar Tarefa}
O objetivo deste caso de uso é permitir que o usuário arquive uma tarefa cadastrada

\subsection{Atores}

\begin{enumerate}
  \item Scrum Master
  \item Diretor de Desenvolvimento
\end{enumerate}

\subsection{Pré-Condições}
\begin{enumerate}
  \item O Scrum Master, ou o Diretor de Desenvolvimento deve estar logado no sistema.
  \item Existir uma tarefa cadastrada dentro de um projeto na base de dados.
\end{enumerate}

\subsection{Fluxo Principal}
O caso de uso começa quando o diretor de desenvolvimento ou o scrum master,visualiza o quadro de tarefas de um projeto

\begin{enumerate}
  \item O diretor de desenvolvimento, ou o scrum master seleciona a opção de arquivar tarefa.
  \item O sistema informa ao diretor de desenvolvimento, ou ao scrum master do arquivamento.
  \item O diretor de desenvolvimento, ou o scrum master é redirecionado a tela de visualização do quadro de tarefas do projeto.
  \item A tarefa não estará mais listada no quadro de tarefas.
  \item O caso de uso é encerrado.
\end{enumerate}

\subsection{Fluxos Alternativos}

\subsubsection{Ativar Tarefa}
O caso de uso começa quando o diretor de desenvolvimento ou o scrum master,visualiza o quadro de tarefas de um projeto.

\begin{enumerate}
  \item O diretor de desenvolvimento ou o scrum master seleciona a opção de reativar tarefas no menu do sistema.
  \item O diretor de desenvolvimento ou o scrum master seleciona uma tarefa a ser ativada.
  \item O sistema informa do reativamento da tarefa.
  \item O diretor de desenvolvimento ou o scrum master confirma o reativamento da tarefa
  \item O diretor de desenvolvimento ou o scrum master é redirecionado a tela de visualização do quadro de tarefas.
  \item O sistema exibe a tarefa reativada na tela de visualização do quadro de tarefas.
  \item O caso de uso é encerrado
\end{enumerate}


\section{Regras de Negócio}

\subsection{RN1}

O quadro de tarefas do projeto deve exibir as tarefas organizadas de acordo com o status das tarefas:
\begin{itemize}
  \item To Do
  \item Doing
  \item Done
\end{itemize}

\subsection{RN2}

As tarefas arquivadas não serão exibidas no quadro de tarefas do projeto. Estarão disponíveis em uma tela separada. [RN5]

\subsection{RN3}

As tarefas possuem as seguintes informações:

\begin{itemize}
  \item Nome
  \begin{enumerate}
    \item Formato: Texto
    \item Tamanho mínimo: 1 caractere
    \item Tamanho máximo: 50 caracteres
  \end{enumerate}
  \item Descrição
  \begin{enumerate}
    \item Formato: Texto
    \item Tamanho mínimo: 1 caractere
    \item Tamanho máximo: 50 caracteres
  \end{enumerate}
  \item Prioridade
  \begin{enumerate}
    \item Opções: Baixa, Média, Alta
  \end{enumerate}
  \item Status
  \begin{enumerate}
    \item Opções: To Do, Doing, Done
  \end{enumerate}
\end{itemize}

\subsection{RN4}

A opção de arquivar é apenas visível e acessível para o diretor de desenvolvimento.

\subsection{RN5}

A tela de projetos arquivados deve exibir todos os projetos ordenados pelo nome de forma ascendente, em uma lista.

\subsection{RN6}

A opção de criar projeto é apenas visível e acessível para o diretor de desenvolvimento.

\subsection{RN7}

Os projetos possuem as seguintes informações:

\begin{itemize}
  \item Nome
  \begin{enumerate}
    \item Formato: Texto
    \item Tamanho mínimo: 1 caractere
    \item Tamanho máximo: 50 caracteres
  \end{enumerate}
  \item Descrição
  \begin{enumerate}
    \item Formato: Texto
    \item Tamanho mínimo: 1 caractere
    \item Tamanho máximo: 50 caracteres
  \end{enumerate}
  \item Cliente
  \begin{enumerate}
    \item Formato: Texto
    \item Tamanho mínimo: 1 caractere
    \item Tamanho máximo: 50 caracteres
  \end{enumerate}
  \item Data Inicial
  \begin{enumerate}
    \item Formato: Data
    \item Não pode ser anterior a data inicial
  \end{enumerate}
  \item Imagem
  \begin{enumerate}
    \item Formato: pnj, jpg
  \end{enumerate}
  \item Desenvolvedores
  \begin{enumerate}
    \item Múltipla escolha: Lista de desenvolvedores da InSoft cadastrados no sistema
  \end{enumerate}
\end{itemize}

\subsection{RN8}

Sempre que uma operação não for concluída com sucesso, o sistema deve informar que a operação não foi efetuada com sucesso e retornar ao passo anterior.
