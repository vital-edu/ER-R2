\chapter{Especificação de Caso de Uso}


\section{UC01 -Visualizar Projeto}

\subsection{Atores}

\subsection{Pré-Condições}

\subsection{Fluxo Principal}

\subsection{Fluxos Alternativos}


\section{UC02 -Editar Projeto}

\subsection{Atores}

\subsection{Pré-Condições}

\subsection{Fluxo Principal}

\subsection{Fluxos Alternativos}



\section{UC03 - Visualizar Projeto}

\subsection{Atores}
\begin{itemize}
  \item Usuário 
\end{itemize}
\subsection{Pré-Condições}

\begin{itemize}
  \item O usuário deve estar logado.
  \item Deve existir um projeto cadastrado na base de dados da aplicação. 
\end{itemize}

\subsection{Fluxo Principal}
O caso de uso começa quando o usuário acessa a tela de seleção de projetos.

\begin{itemize}
  \item O usuário seleciona a opção de visualizar o quadro de tarefas de um projeto dentre os listados.
  \item O sistema retorna ao usuário o quadro de tarefas do projeto [RN1] [RN2].
  \item O caso de uso é encerrado. 
\end{itemize}

\subsection{Fluxos Alternativos}

\subsubsection{Visualizar informações do projeto}
Este fluxo se inicia no passo 1 do fluxo principal quando o usuário ao invés de selecionar a opção de visualizar o quadro de tarefas de um projeto, seleciona a opção de visualizar as informações de um projeto

\begin{itemize}
  \item O sistema retorna ao usuário as informações do sistema [RN3]
  \item O caso de uso é encerrado
\end{itemize}


