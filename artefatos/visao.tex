\chapter{Documento de Visão}
\label{sec:art-visao}
\section{Histórico da Revisão}

\section{Introdução}

A finalidade deste documento é coletar, analisar e definir necessidades e recursos de nível superior do Sistema de Gestão de Projetos de Desenvolvimento da inSoft (SGPDi). Ao longo deste documento, serão apresentados os problemas da inSoft que motivaram o desenvolvimento da solução proposta, os envolvidos e afetados pela solução, as necessidades dos envolvidos que precisam ser satisfeitas pelo sistema, a visão geral da solução, bem como seus recursos.

Assim, propõe-se que ao ler este documento, haverá o total entendimento sobre os motivos que levaram à proposta do desenvolvimento do SGPDi, bem como as funcionalidades que esse sistema irá fornecer, definindo-se o escopo do projeto.

  % \section{Referências}

  \section{Posicionamento}
    \subsection{Descrição do Problema}

      \begin{table}[H]
        \centering
        \begin{tabular}{|>{\columncolor[HTML]{C0C0C0}}p{0.26\textwidth}|p{0.65\textwidth}|}
          \hline
          O problema de         &   falta de controle do processo de desenvolvimento \\ \hline
          afeta                 &   toda a equipe de desenvolvimento                 \\ \hline
          cujo impacto é        &   a desorganização da equipe de desenvolvimento;
                                    o desconhecimento das tarefas a serem executadas pela equipe;
                                    a demora na execução dos projetos de desenvolvimento e eventual cancelamento dos mesmos                                         \\ \hline
          uma boa solução seria &   desenvolver um sistema em \emph{software} que possibilite:
                                    disponibilizar todas as informação dos membros relevantes para o desenvolvimento;
                                    gerenciar múltiplos projetos de desenvolvimento;
                                    gerenciar as tarefas de desenvolvimento de cada projeto;
                                    \\ \hline
        \end{tabular}
        \caption{Descrição do Problema da inSoft}
      \end{table}

    \subsection{Necessidades}

      Dessa forma, a partir do problema elicitado, pode-se extrair as seguintes necessidades:

        \begin{itemize}
          \item Permitir o gerenciamento de membros;
          \item Visualizar as habilidades técnicas de cada membro;
          \item Visualizar os horários disponíveis de cada membro;
          \item Permitir o gerenciamento de múltiplos projetos;
          \item Permitir que cada projeto tenha suas próprias atividades e participantes;
          \item Visualizar as tarefas de desenvolvimento de cada projeto;
          \item Visualizar o \emph{status} de cada tarefa de desenvolvimento;
          \item Designar responsáveis para cada tarefas de desenvolvimento.
        \end{itemize}

    \subsection{Sentença de Posição do Produto}

      \begin{table}[H]
        \centering
        \begin{tabular}{|>{\columncolor[HTML]{C0C0C0}}p{0.20\textwidth}|p{0.71\textwidth}|}
          \hline
          Para            &   a inSoft                                                 \\ \hline
          Que             &   necessita de um sistema que a auxilie no melhoramento de seu processo de desenvolvimento
                                                                                       \\ \hline
          O SGi           &   é um sistema gerenciador de projetos de desenvolvimento  \\ \hline
          Que             &   permite o monitoramento das atividades de desenvolvimento de cada projeto da empresa, bem como da equipe de desenvolvimento envolvida em cada projeto                \\ \hline
          Ao contrário do &   Trello                                                   \\ \hline
          Nosso produto   &   permite uma melhor visualização das informações de cada membro, e das atividades que cada um desempenha, possibilitando maior controle sobre a equipe.               \\ \hline
        \end{tabular}
        \caption{Posição do SGi}
      \end{table}

  \section{Descrições dos Envolvidos e dos Usuários}
    \subsection{Resumo dos Envolvidos}

     \begin{table}[]
\centering
\caption{Resumo dos Envolvidos}
\label{tab:res-envolvido}
\begin{tabular}{|l|l|l|}
\hline
\rowcolor[HTML]{C0C0C0} 
Nome                                                                    & Descrição                                                                                                                                   & Responsabilidades                                                                                                                                                                                                                                             \\ \hline
\begin{tabular}[c]{@{}l@{}}Diretor de Gestão \\ de Pessoas\end{tabular} & \begin{tabular}[c]{@{}l@{}}Funcionário da inSoft que \\ dirige o setor de recursos \\ humanos da InSoft.\end{tabular}                       & \begin{tabular}[c]{@{}l@{}}Assegura que todos os \\ contratados pela InSoft \\ estarão com seus dados, \\ horários de trabalho e \\ habilidades específicas \\ atualizados.\end{tabular}                                                                      \\ \hline
\begin{tabular}[c]{@{}l@{}}Diretor de \\ Desenvolvimento\end{tabular}   & \begin{tabular}[c]{@{}l@{}}Funcionário da inSoft que\\  dirige os projetos de \\ desenvolvimento da Empresa\end{tabular}                    & \begin{tabular}[c]{@{}l@{}}Assegura que todos os\\  envolvidos em tal projeto \\ estão aptos a tal sistema \\ que será\end{tabular}                                                                                                                           \\ \hline
Desenvolvedor                                                           & \begin{tabular}[c]{@{}l@{}}Qualquer funcionário da \\ InSoft integrante de algum\\  projeto de desenvolvimento\\  de software.\end{tabular} & \begin{tabular}[c]{@{}l@{}}Desenvolve as soluções \\ em software propostas \\ aos clientes da inSoft.\end{tabular}                                                                                                                                            \\ \hline
Negociador                                                              & \begin{tabular}[c]{@{}l@{}}Funcionário da InSoft \\ que negocia com o cliente.\end{tabular}                                                 & \begin{tabular}[c]{@{}l@{}}Cria protótipos da solução\\  proposta, confecciona \\ e firma contratos com o \\ cliente.Documenta as \\ especificações iniciais da \\ solução em software\end{tabular}                                                           \\ \hline
Scrum Master                                                            & \begin{tabular}[c]{@{}l@{}}Desenvolvedor que \\ coordenar um time \\ de desenvolvimento.\end{tabular}                                       & \begin{tabular}[c]{@{}l@{}}Soluciona problemas e \\ riscos que possam vir a \\ ocorrer durante o \\ desenvolvimento\\ Assegura queo time \\ está atendendo os prazos \\ estipulados para cada \\ atividade proposta no \\ cronograma do Projeto.\end{tabular} \\ \hline
\end{tabular}
\end{table} 

    \subsection{Resumo dos Usuários}

      \begin{table}[H]
        \centering
        \begin{tabular}{|p{0.20\textwidth}|p{0.23\textwidth}|p{0.23\textwidth}|p{0.23\textwidth}|}
          \hline
          \rowcolor[HTML]{C0C0C0}
          \multicolumn{1}{c}{Nome} & \multicolumn{1}{|c|}{Descrição} & \multicolumn{1}{|c|}{Responsabilidade} & \multicolumn{1}{|c|}{Envolvido} \\ \hline
            Desenvolvedores
              & Qualquer funcionário da Insoft que possui conhecimentos de alguma linguagem e com horário disponível para trabalhar em algum projeto, ou já trabalha em algum projeto.
              & Analisar se todas suas informações estão corretas e atualizar suas informações relacionadas a horário e quadro de habilidades caso seja necessário.
              & Atualização dos registros, mediante permissão. \\ \hline
            \emph{Scrum Master}
              & Tem o papel de técnico do time e assim orientar o time na realização de seus trabalhos assim os auxiliando a seguir as práticas ágeis adotadas pela InSoft.
              & Verificar os horários dos desenvolvedores de certo projeto.
              & Envolvido na parte de manter maior controle da equipe a qual está destinado a ajudar. \\ \hline
            Diretor de Desenvolvimento
              & Tem o papel de liderança em torno de todas as equipes de desenvolvimento da InSoft.
              & Designa funcionários da InSoft para atuarem como desenvolvedores de acordo com suas competências e horários disponíveis;
                Atualiza os horários dos funcionários após serem realocados para algum projeto.
              & Envolvido com a seleção e distribuição de desenvolvedores para projetos seguindo os horários e habilidades de cada funcionário. \\ \hline
            Diretor de Gestão de Pessoas
              & Diretor a qual fica a cargo de gerenciar os recursos humanos da InSoft.
              & Adicionar novos funcionários no sistema atendendo suas habilidades e horário;
                Retirar registro dos funcionários que foram desligados da empresa.
              & Envolvido com a área de seleção e cadastro de cada novo funcionário da InSoft, sendo também responsável pela área de desligamento e remoção de cadastros. \\ \hline
        \end{tabular}
        \caption{Resumo dos usuários}
      \end{table}

    \subsection{Ambiente do Usuário}

      A inSoft possui XX funcionários, sendo que em cada projeto de desenvolvimento pretende-se alocar 4 funcionários. Atualmente não há dados sobre o tempo de execução de tarefas da área de desenvolvimento devido ao pouco tempo que a empresa encontra-se em operação, e por não haver concluído nenhum projeto de desenvolvimento até o presente momento.

      Embora a inSoft utilize atualmente o Trello no gerenciamento de suas atividades, o mesmo não consegue cumprir com as necessidades da empresa, devendo ser substituido pelo SGi. Também são utilizados documentos eletrônicos para o gerenciamento de alguns setores da empresa. O setor de Recursos Humanos, por exemplo, utiliza planilhas eletrônicas para o gerenciamento das informações sobre os membros da inSoft, e por conta disto é necessários que se desenvolva uma funcionalidade que permita a importação desses dados gravados em documentos eletrônicos.

      Por não possuir ambiente físico estável, ou equipamentos eletrônicos próprios, a inSoft prefere trabalhar com sistemas de \emph{software} \emph{online}.

    \subsection{Resumo das Principais Necessidades dos Envolvidos ou dos Usuários}

      As atuais  ferramentas utilizadas no gerenciamento da equipe de desenvolvimento provocam os seguintes problemas:

      \begin{itemize}
        \item Falta de visibilidade das habilidades dos funcionários
        \item Falta de visibilidade da disponibilidade de horário dos funcionários
        \item Falta de visibilidade das tarefas alocadas
        \item Falta de visibilidade dos responsáveis por cada tarefa
        \item Falta de visibilidade quanto a documentação produzida pela equipe de desenvolvimento
        \item Dificuldade de gerir os documentos produzidos pela equipe de desenvolvimento
        \item Dificuldade de gerir os responsáveis por cada atividade
        \item Dificuldade de gerir a alocação de tarefas
      \end{itemize}

      Todos os problemas supracitados são causados pela dificuldade em convergir todas as informações registradas pela inSoft tanto em documentos eletrônicos, como em ferramentas de gerenciamento de projeto que utilizam.

      Embora a inSoft tente estabelecer uma metodologia de organização sistemática, a falta de centralização de todas as informações importantes para o gerenciamento dos projetos de desenvolvimento, bem como da equipe de desenvolvimento, tem frustado suas tentativas de contornar esses problemas.

      A fim de resolver esses problemas, a inSoft espera que o SGi possibilite, de forma centralizada, o controle sobre as informações dos funcionários, a disponibilidades dos mesmos; a organização dos projetos de desenvolvimento e suas atividades; e a possibilidade de conseguir alocar facilmente funcionários para um determinado projeto e atividade sem causar conflitos de disponibilidade de horários.

\begin{table}[]
\centering
\label{tab:nec-prob}
\begin{tabular}{|c|c|c|c|c|}
\hline
\rowcolor[HTML]{C0C0C0} 
Necessidade                                                                                                                                  & Prioridade & Preocupações                                                                                                                                                                         & Solução Atual                                                                                                                                            & Soluções Propostas                                                                                                                                                                          \\ \hline
\begin{tabular}[c]{@{}l@{}}Permitir o \\ gerenciamento \\ de funcionários\end{tabular}                                                       & Média      & \begin{tabular}[c]{@{}l@{}}Limitar o \\ acesso \\ dos funcionários \\ aos projetos dos\\  quais participam\end{tabular}                                                              & \begin{tabular}[c]{@{}l@{}}Os funcionários \\ são gerenciados \\ pela equipe de \\ RH através de \\ planilhas \\ eletrônicas\end{tabular}                & \begin{tabular}[c]{@{}l@{}}Transferência das \\ informações dos \\ funcionários \\ mantidas em \\ planilhas eletrônicas \\ para o sistema, \\ permitindo um \\ melhor controle\end{tabular} \\ \hline
\begin{tabular}[c]{@{}l@{}}Visualizar as \\ habilidades \\ técnicas \\ e os horários \\ disponíveis de \\ cada \\ desenvolvedor\end{tabular} & Alta       & \begin{tabular}[c]{@{}l@{}}Não há um \\ controle \\ eficiente\\ sobre as \\ habilidades\\ técnicas e os \\ horários de \\ trabalho\\ são \\ monitorados\\ informalmente\end{tabular} & \begin{tabular}[c]{@{}l@{}}Gerenciamento \\ das habilidades é \\ feito através de \\ planilhas \\ eletrônicas\\ disponibilizadas \\ pelo RH\end{tabular} & \begin{tabular}[c]{@{}l@{}}Acesso \\ on-line aos \\ horários de \\ trabalho \\ para \\ visualização e \\ atualização \\ simplificada\end{tabular}                                           \\ \hline
\begin{tabular}[c]{@{}l@{}}Permitir o \\ gerenciamento \\ de \\ múltiplos \\ projetos\end{tabular}                                           & Média      & \begin{tabular}[c]{@{}l@{}}Falta de \\ visibilidade\\  dos projetos \\ em andamento\end{tabular}                                                                                     & \begin{tabular}[c]{@{}l@{}}Os projetos \\ são\\  gerenciados \\ através\\  de um kanban \\ online (Trello)\end{tabular}                                  & \begin{tabular}[c]{@{}l@{}}Disponibilização \\ de \\ lista de \\ projetos \\ criados\end{tabular}                                                                                           \\ \hline
\begin{tabular}[c]{@{}l@{}}Gerenciar \\ tarefas \\ de \\ desenvolvimento \\ de cada projeto\end{tabular}                                     & Média      & \begin{tabular}[c]{@{}l@{}}Pouca \\ visibilidade \\ das atividades \\ desempenhadas \\ por cada \\ desenvolvedor\end{tabular}                                                        & \begin{tabular}[c]{@{}l@{}}As atividades \\ sãogerenciadas \\ através \\ de um kanban \\ online (Trello)\end{tabular}                                    & \begin{tabular}[c]{@{}l@{}}Informação \\ sobre \\ todas as \\ tarefas \\ executadas \\ pelo \\ desenvolvedor\end{tabular}                                                                   \\ \hline
\end{tabular}
\caption{Tabela de Necessidades}
\end{table}

\subsubsection{Prioridade}
\begin{itemize}
  \item Alta: essa necessidade é fundamental à resolução do problema.
  \item Média: essa necessidade é importante, porém a empresa já possui paliativos.
  \item Baixa: essa necessidade não é prioritária, mas agrega valor ao usuário.
\end{itemize}

    \subsection{Alternativas e Concorrência}
       \label{subsec:Concorrentes}
      \begin{table}[H]
        \centering
        \begin{tabular}{|p{0.1\textwidth}|p{0.26\textwidth}|p{0.26\textwidth}|p{0.26\textwidth}|}
          \hline
          \rowcolor[HTML]{C0C0C0}
          \multicolumn{1}{c}{Produto} & \multicolumn{1}{|c|}{Afeta qual problema ?} & \multicolumn{1}{|c|}{Pontos fortes} & \multicolumn{1}{|c|}{Pontos Fracos} \\ \hline
            Nex
              & Falta de visibilidade de informações de cada membro da equipe.
              & Sistema de cadastro simples;
                Pode ser usado na parte de controle de finanças da InSoft.
              & Muito genérico e não atende todas as necessidades esperadas para resolver o problema, como mostrar horários disponíveis e o quadro de habilidades de cada usuário. \\ \hline
            Trello
              & Membros não têm visibilidade dos processos da empresa.
              & Mostra todos as atividades para a equipe de desenvolvimento sobre o como está o andamento do projeto de tal Software.
              & Não integra tudo que o sistema necessita em relação ao desenvolvimento de software, mas sim um Kanban simples com o fluxo de atividades;
                Necessita cadastro e não possui modo de exportar, fazendo assim a empresa refém do serviço que pode deixar de ser prestado a qualquer momento. \\ \hline
            Arivo
              & Falta de visibilidade de informações de cada membro da equipe;
                Membros não têm visibilidade dos processos da empresa.
              & Consegue unir a parte de cadastro de clientes, junto com uma área de modelo de processo, assim podendo resolver dois problemas em um.
              & Pago;
                Criado originalmente para CRM (Customer Relationship Management), ferramenta não atrelada com a função base a qual seria inserida na InSoft, podendo assim causar certas inconsistências. \\ \hline
            Github
            +
            Zenhub
              & Visibilidade de tarefas, status da tarefa e responsável pela 
                tarefa. Ainda auxilia o GCS da empresa.
              & Fica semelhante ao Trello. Tem a grande vantagem de unir 
                visibilidade de código e controle de mudanças à gerência de 
                tarefas, responsabilidades e issues.
              & Funciona somente no chrome. Não atende as necessidades de 
                habilidades técnicas e de horários disponíveis dos membros. \\ \hline
        \end{tabular}
        \caption{Alternativas e concorrência}
      \end{table}

  \section{Visão Geral do Produto}

    \subsection{Perspectiva do Produto}

      O SGPDi tem como meta resolver os problemas de falta de controle dos processos, referentes a área de desenvolvimento dentro da empresa inSoft, fornecendo um meio de facilitar e organizar a visibilidade desses processos. Funciona como uma ferramenta de gestão para manter e registrar as informações referentes a disponibilidade de horários e habilidades especificas de cada membro da empresa e alocá-los às suas devidas tarefas e atividades, nas quais são mais justas de acordo com suas informações registradas na ferramenta.
      
      Como já foi citado, essa ferramenta provê uma melhor visibilidade do processo afim de servir como meio de auxílio da parte de gestão da empresa, que além de se evitar possíveis falhas no controle do processo, aumentaria a eficiência e consequentemente a produtividade dentro dos processos da empresa.


    \subsection{Suposições e Dependências}
        
    \begin{table}[H]
    \centering
    \caption{Fatores e Motivos para mudança}
    \label{tab:fatores}
    \begin{tabular}{|l|l|}
    \hline
    \rowcolor[HTML]{C0C0C0} 
    Fator                                                                                                                                                                       & Motivos plausíveis para mudança                                                                                                                                                                                                                                                                                                                                                              \\ \hline
    \begin{tabular}[c]{@{}l@{}}A solução deve ser \\ desenvolvida para \\ ser uma aplicação \\ Web.\end{tabular}                                                                & \begin{tabular}[c]{@{}l@{}}Até o presente momento todas as ideias se adequam a uma \\ solução web,já que seria de fácil acesso a todos os \\ membros (que poderão ser usuários do sistema) pela \\ InSoft. Caso não seja possível encontrar um servidor para\\  ser utilizado como host o visão deverá ser alterado.\end{tabular}                                                            \\ \hline
    \begin{tabular}[c]{@{}l@{}}InSoft aderir ao \\ uso de algum outro \\ software concorrente \\ que seja capaz de \\ resolver um dos \\ problemas \\ encontrados.\end{tabular} & \begin{tabular}[c]{@{}l@{}}Até o momento vários problemas foram encontrados para \\ serem abordados pelo software em questão. A InSoft \\ não se pronunciou de usar algum software concorrente \\ para suprir algumas das necessidades que ajudem \\ a realizar este problema, caso cheguem a aderir o visão \\ deverá ser alterado possuindo um escopo com fronteira \\ menor.\end{tabular} \\ \hline
    \end{tabular}
    \end{table}

  \section{Recursos do Produto}

  \subsection{Requisitos Funcionais}

  \begin{table}[]
\centering
\caption{Tabela de Requisitos Funcionais}
\label{tab:req-fun-vis}
\begin{tabular}{|l|l|}
\hline
RF1  & Permitir o cadastro de desenvolvedores                                                                                                                \\ \hline
RF2  & \begin{tabular}[c]{@{}l@{}}Permitir o envio de um arquivo digital específico com as informações dos \\ desenvolvedor a serem cadastrados\end{tabular} \\ \hline
RF3  & Permitir a edição das informações dos desenvolvedores                                                                                                 \\ \hline
RF4  & Permitir a alteração das informações dos usuários                                                                                                     \\ \hline
RF5  & Permitir o desativamento de usuários                                                                                                                  \\ \hline
RF6  & Permitir o reativamento de usuários                                                                                                                   \\ \hline
RF7  & Permitir a visualização dos perfis                                                                                                                    \\ \hline
RF8  & Permitir a visualização das habilidades especificas dos desenvolvedores                                                                               \\ \hline
RF9  & Permitir a visualização de horários dos desenvolvedores                                                                                               \\ \hline
RF10 & Fornecer filtros de pesquisa para as informações dos usuários                                                                                         \\ \hline
RF11 & Permitir a criação de projetos                                                                                                                        \\ \hline
RF12 & Permitir a visualização das tarefas dos projetos                                                                                                      \\ \hline
RF13 & Permitir a busca de projetos                                                                                                                          \\ \hline
RF14 & Fornecer filtros na visualização dos projetos                                                                                                         \\ \hline
RF15 & \begin{tabular}[c]{@{}l@{}}O sistema deve gerar automaticamente o status do projeto a partir dos \\ status das tarefas do projeto\end{tabular}        \\ \hline
RF16 & Permitir o arquivamento de projetos.                                                                                                                  \\ \hline
RF17 & Permitir a alteração dos dados do projeto.                                                                                                            \\ \hline
RF18 & Reorganizar o status do projeto a partir das alterações no mesmo.                                                                                     \\ \hline
RF19 & Permitir a inserção de usuários em projetos                                                                                                           \\ \hline
RF20 & Permitir a escolha de um scrum master do projeto.                                                                                                     \\ \hline
RF21 & Permitir a alteração do scrum master do projeto.                                                                                                      \\ \hline
RF22 & Permitir o cadastro de tarefas pertencentes à um projeto                                                                                              \\ \hline
RF23 & Permitir a atribuição de um único responsável para cada tarefa                                                                                        \\ \hline
RF24 & Permitir a inserção dos desenvolvedores à tarefa.                                                                                                     \\ \hline
RF25 & Exibir o responsável da tarefa                                                                                                                        \\ \hline
RF26 & Exibir os desenvolvedores que estão executando a tarefa -26                                                                                           \\ \hline
RF27 & Permitir a visualização do status das tarefas.                                                                                                        \\ \hline
RF28 & Permitir a alteração do status das tarefas do projeto.                                                                                                \\ \hline
RF29 & Permitir a visualização dos responsáveis de uma tarefa.                                                                                               \\ \hline
RF30 & Permitir que o scrum master crie relatórios sobre cada sprint                                                                                         \\ \hline
RF31 & Permitir que o diretor de desenvolvimento aprove o relatório                                                                                          \\ \hline
RF32 & Permitir que o diretor de desenvolvimento rejeite o relatório                                                                                         \\ \hline
RF33 & Permitir a exibição das tarefas dos projetos                                                                                                          \\ \hline
RF34 & Permitir a busca por tarefas dos projetos                                                                                                             \\ \hline
RF35 & Permitir o uso de filtros na visualização das tarefas dos projetos                                                                                    \\ \hline
\end{tabular}
\end{table}

 \subsection{Requisitos Não-Funcionais}

\begin{table}[]
\centering
\label{tab:req-n-fun-vis}
\begin{tabular}{|l|l|l|l|}
\hline
\rowcolor[HTML]{C0C0C0} 
ID    & Tipo            & Característica         & Descrição                                                                                                                                            \\ \hline
RNF01 & Usabilidade     & Erros                  & \begin{tabular}[c]{@{}l@{}}O sistema deve exibir mensagens de\\  erros claras e informativas\end{tabular}                                            \\ \hline
RNF02 & Usabilidade     & Interface de Usuário   & \begin{tabular}[c]{@{}l@{}}O sistema deve possuir uma interface\\  de usuário organizada e clara\end{tabular}                                        \\ \hline
RNF03 & Usabilidade     & Memorização            & \begin{tabular}[c]{@{}l@{}}O sistema possuirá elementos visuais \\ que facilitem a reutilização do sistema.\end{tabular}                             \\ \hline
RNF04 & Suportabilidade & Responsividade         & \begin{tabular}[c]{@{}l@{}}O sistema deve se ajustar a diferentes\\  tamanhos e resoluções de tela\end{tabular}                                      \\ \hline
RNF05 & Suportabilidade & Portabilidade          & \begin{tabular}[c]{@{}l@{}}O sistema deve estar acessível em \\ todos os navegadores de internet.\end{tabular}                                       \\ \hline
RNF06 & Confiabilidade  & Funcionamento          & \begin{tabular}[c]{@{}l@{}}O sistema deve ter disponibilidade de\\  98\% do tempo\end{tabular}                                                       \\ \hline
RNF07 & Performance     & Capacidade de usuários & \begin{tabular}[c]{@{}l@{}}O sistema deve permitir 100 usuários\\  simultâneos\end{tabular}                                                          \\ \hline
RNF08 & Performance     & Tempo de resposta      & \begin{tabular}[c]{@{}l@{}}O sistema deve responder a uma \\ requisição em menos de 0,5 segundo\end{tabular}                                         \\ \hline
RNF09 & Performance     & Leitura de dados       & \begin{tabular}[c]{@{}l@{}}O sistema deve reconhecer arquivos\\  com extensão .XML\end{tabular}                                                      \\ \hline
RNF10 & Segurança       & Acesso                 & \begin{tabular}[c]{@{}l@{}}O sistema deve restringir certas funções\\  de acordo com os níveis de permissões\\  de acessos dos usuários\end{tabular} \\ \hline
\end{tabular}
\caption{Requisitos Não-Funcionais}
\end{table}
  \section{Outros Requisitos do Produto}
