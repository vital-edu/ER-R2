\chapter{Documento de Visão}

  \section{Histórico da Revisão}

  \begin{table}[H]
    \centering
    \begin{tabular}{|l|l|l|l|}
      \hline
      \multicolumn{1}{|c|}{Data} & \multicolumn{1}{c|}{Vesão} & \multicolumn{1}{c|}{Descrição}     & \multicolumn{1}{c|}{Autor} \\ \hline
      06/11/2015                & 0.1                         & Incluído posicionamento do projeto & Eduardo Vital       \\ \hline
    \end{tabular}
    \caption{Descrição do Problema}
  \end{table}

  \section{Introdução}

    A finalidade deste documento é coletar, analisar e definir necessidades e recursos de nível superior do Sistema de Gerenciamento da inSoft (SGi). Ao longo deste documento, serão apresentados os problemas na inSoft que motivaram o desenvolvimento da solução proposta, os envolvidos e afetados pela solução, as necessidades dos envolvidos que precisam ser satisfeitas pelo sistema, a visão geral da solução, bem como seus recursos.

    Assim, propõe-se que ao ler este documento, haverá o total entendimento sobre os motivos que levaram à proposta do desenvolvimento do SGi, bem como as funcionalidades que esse sistema irá fornecer, dfinindo-se o escopo do projeto.

  % \section{Referências}

  \section{Posicionamento}
    \subsection{Descrição do Problema}

      \begin{table}[H]
        \centering
        \begin{tabular}{|>{\columncolor[HTML]{C0C0C0}}p{0.25\textwidth}|p{0.75\textwidth}|}
          \hline
          O problema de         &   falta de controle do processo de desenvolvimento \\ \hline
          afeta                 &   toda a equipe de desenvolvimento                 \\ \hline
          cujo impacto é        &   a desorganização da equipe de desenvolvimento;
                                    o desconhecimento das tarefas a serem executadas pela equipe;
                                    a demora na execução dos projetos de desenvolvimento e eventual cancelamento dos mesmos                                         \\ \hline
          uma boa solução seria &   desenvolver um sistema em \emph{software} que possibilite:
                                    disponibilizar todas as informação dos membros relevantes para o desenvolvimento;
                                    gerenciar múltiplos projetos de desenvolvimento;
                                    gerenciar as tarefas de desenvolvimento de cada projeto;
                                    \\ \hline
        \end{tabular}
        \caption{Descrição do Problema da inSoft}
      \end{table}

    \subsection{Necessidades}

      Dessa forma, a partir do problema elicitado, pode-se extrair as seguintes necessidades:

        \begin{itemize}
          \item Permitir o gerenciamento de membros;
          \item Visualizar as habilidades técnicas de cada membro;
          \item Visualizar os horários disponíveis de cada membro;
          \item Permitir o gerenciamento de múltiplos projetos;
          \item Permitir que cada projeto tenha suas próprias atividades e participantes;
          \item Visualizar as tarefas de desenvolvimento de cada projeto;
          \item Visualizar o \emph{status} de cada tarefa de desenvolvimento;
          \item Designar responsáveis para cada tarefas de desenvolvimento.
        \end{itemize}

    \subsection{Sentença de Posição do Produto}

      \begin{table}[H]
        \centering
        \begin{tabular}{|>{\columncolor[HTML]{C0C0C0}}p{0.25\textwidth}|p{0.75\textwidth}|}
          \hline
          Para            &   a inSoft                                                \\ \hline
          Que             &   necessita de um sistema que a auxilie no melhoramento de seu processo de desenvolvimento
                                                                                       \\ \hline
          O SGi           &   é um sistema gerenciador de projetos de desenvolvimento  \\ \hline
          Que             &   permite o monitoramento das atividades de desenvolvimento de cada projeto da empresa, bem como da equipe de desenvolvimento envolvida em cada projeto                \\ \hline
          Ao contrário do &   Trello \\ \hline
          Nosso produto   &   permite uma melhor visualização das informações de cada membro, e das atividades que cada um desempenha, possibilitando maior controle sobre a equipe.                \\ \hline
        \end{tabular}
        \caption{Posição do SGi}
      \end{table}

  \section{Descrições dos Envolvidos e dos Usuários}
    \subsection{Resumo dos Envolvidos}
    \subsection{Resumo dos Usuários}
    \subsection{Ambiente do Usuário}
    \subsection{Resumo das Principais Necessidades dos Envolvidos ou dos Usuários}
    \subsection{Alternativas e Concorrência}
    \section{Visão Geral do Produto}
    \subsection{Perspectiva do Produto}
    \subsection{Suposições e Dependências}
  \section{Recursos do Produto}
  \section{Outros Requisitos do Produto}
