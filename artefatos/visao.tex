\chapter{Documento de Visão}
\label{sec:art-visao}
  \section{Histórico da Revisão}

  \begin{table}[H]
    \centering
    \begin{tabular}{|l|l|l|l|}
      \hline
      \multicolumn{1}{|c|}{Data} & \multicolumn{1}{c|}{Vesão} & \multicolumn{1}{c|}{Descrição}     & \multicolumn{1}{c|}{Autor} \\ \hline
      06/11/2015                & 0.1                         & Incluído posicionamento do projeto & Eduardo Vital       \\ \hline
      06/11/2015                & 0.2                         & Incluído informação dos usuários e envolvidos & Nicolas  Oliveira
        \\ \hline
    \end{tabular}
    \caption{Descrição do Problema}
  \end{table}

  \section{Introdução}

    A finalidade deste documento é coletar, analisar e definir necessidades e recursos de nível superior do Sistema de Gerenciamento da inSoft (SGi). Ao longo deste documento, serão apresentados os problemas na inSoft que motivaram o desenvolvimento da solução proposta, os envolvidos e afetados pela solução, as necessidades dos envolvidos que precisam ser satisfeitas pelo sistema, a visão geral da solução, bem como seus recursos.

    Assim, propõe-se que ao ler este documento, haverá o total entendimento sobre os motivos que levaram à proposta do desenvolvimento do SGi, bem como as funcionalidades que esse sistema irá fornecer, dfinindo-se o escopo do projeto.

  % \section{Referências}

  \section{Posicionamento}
    \subsection{Descrição do Problema}

      \begin{table}[H]
        \centering
        \begin{tabular}{|>{\columncolor[HTML]{C0C0C0}}p{0.26\textwidth}|p{0.65\textwidth}|}
          \hline
          O problema de         &   falta de controle do processo de desenvolvimento \\ \hline
          afeta                 &   toda a equipe de desenvolvimento                 \\ \hline
          cujo impacto é        &   a desorganização da equipe de desenvolvimento;
                                    o desconhecimento das tarefas a serem executadas pela equipe;
                                    a demora na execução dos projetos de desenvolvimento e eventual cancelamento dos mesmos                                         \\ \hline
          uma boa solução seria &   desenvolver um sistema em \emph{software} que possibilite:
                                    disponibilizar todas as informação dos membros relevantes para o desenvolvimento;
                                    gerenciar múltiplos projetos de desenvolvimento;
                                    gerenciar as tarefas de desenvolvimento de cada projeto;
                                    \\ \hline
        \end{tabular}
        \caption{Descrição do Problema da inSoft}
      \end{table}

    \subsection{Necessidades}

      Dessa forma, a partir do problema elicitado, pode-se extrair as seguintes necessidades:

        \begin{itemize}
          \item Permitir o gerenciamento de membros;
          \item Visualizar as habilidades técnicas de cada membro;
          \item Visualizar os horários disponíveis de cada membro;
          \item Permitir o gerenciamento de múltiplos projetos;
          \item Permitir que cada projeto tenha suas próprias atividades e participantes;
          \item Visualizar as tarefas de desenvolvimento de cada projeto;
          \item Visualizar o \emph{status} de cada tarefa de desenvolvimento;
          \item Designar responsáveis para cada tarefas de desenvolvimento.
        \end{itemize}

    \subsection{Sentença de Posição do Produto}

      \begin{table}[H]
        \centering
        \begin{tabular}{|>{\columncolor[HTML]{C0C0C0}}p{0.20\textwidth}|p{0.71\textwidth}|}
          \hline
          Para            &   a inSoft                                                 \\ \hline
          Que             &   necessita de um sistema que a auxilie no melhoramento de seu processo de desenvolvimento
                                                                                       \\ \hline
          O SGi           &   é um sistema gerenciador de projetos de desenvolvimento  \\ \hline
          Que             &   permite o monitoramento das atividades de desenvolvimento de cada projeto da empresa, bem como da equipe de desenvolvimento envolvida em cada projeto                \\ \hline
          Ao contrário do &   Trello                                                   \\ \hline
          Nosso produto   &   permite uma melhor visualização das informações de cada membro, e das atividades que cada um desempenha, possibilitando maior controle sobre a equipe.               \\ \hline
        \end{tabular}
        \caption{Posição do SGi}
      \end{table}

  \section{Descrições dos Envolvidos e dos Usuários}
    \subsection{Resumo dos Envolvidos}

      \begin{table}[H]
        \centering
        \begin{tabular}{|p{0.25\textwidth}|p{0.25\textwidth}|p{0.41\textwidth}|}
          \hline
          \rowcolor[HTML]{C0C0C0}
          \multicolumn{1}{c}{Nome} & \multicolumn{1}{|c|}{Descrição} & \multicolumn{1}{|c|}{Responsabilidade} \\ \hline
            Diretor de Gestão de Pessoas & Diretor a qual fica a cargo de gerenciar os recursos humanos da InSoft.
                                         & Assegura que todos os contratados pela InSoft estarão com seus dados sobre horários disponíveis e habilidades específicas. \\ \hline
            Diretor de Desenvolvimento   & Diretor a qual fica a cargo a parte de criação de protótipos e gestão de desenvolvedores para certo projeto. & Assegura que todos os envolvidos em tal projeto estão aptos a tal sistema que será feito e se possuem tempo para trabalhar no desenvolvimento. \\ \hline
            Desenvolvedor                & Qualquer funcionário da InSoft destinado a algum projeto de desenvolvimento de software.                  & Assegura que todos seus dados estão corretos e de atualizar conforme a melhora em seu quadro de habilidades e horários. \\ \hline
            Negociador                   & Funcionário da InSoft responsável pelo acordo com o cliente de um projeto.
                                         & Assegura que todas as especificações iniciais estejam dentro de uma lista de habilidades que podem ser feitas pelos desenvolvedores da InSoft. \\ \hline
            \emph{Scrum Master}          & Funcionário da InSoft responsável por um time de desenvolvimento e verificar seu andamento em relação as atividades à serem feitas.
                                         & Assegura que o time está atendendo os prazos estipulados para cada atividade proposta no cronograma do Projeto. \\ \hline
        \end{tabular}
        \caption{Resumo dos envolvidos}
      \end{table}

    \subsection{Resumo dos Usuários}

      \begin{table}[H]
        \centering
        \begin{tabular}{|p{0.20\textwidth}|p{0.23\textwidth}|p{0.23\textwidth}|p{0.23\textwidth}|}
          \hline
          \rowcolor[HTML]{C0C0C0}
          \multicolumn{1}{c}{Nome} & \multicolumn{1}{|c|}{Descrição} & \multicolumn{1}{|c|}{Responsabilidade} & \multicolumn{1}{|c|}{Envolvido} \\ \hline
            Desenvolvedores
              & Qualquer funcionário da Insoft que possui conhecimentos de alguma linguagem e com horário disponível para trabalhar em algum projeto, ou já trabalha em algum projeto.
              & Analisar se todas suas informações estão corretas e atualizar suas informações relacionadas a horário e quadro de habilidades caso seja necessário.
              & Atualização dos registros, mediante permissão. \\ \hline
            \emph{Scrum Master}
              & Tem o papel de técnico do time e assim orientar o time na realização de seus trabalhos assim os auxiliando a seguir as práticas ágeis adotadas pela InSoft.
              & Verificar os horários dos desenvolvedores de certo projeto.
              & Envolvido na parte de manter maior controle da equipe a qual está destinado a ajudar. \\ \hline
            Diretor de Desenvolvimento
              & Tem o papel de liderança em torno de todas as equipes de desenvolvimento da InSoft.
              & Designa funcionários da InSoft para atuarem como desenvolvedores de acordo com suas competências e horários disponíveis;
                Atualiza os horários dos funcionários após serem realocados para algum projeto.
              & Envolvido com a seleção e distribuição de desenvolvedores para projetos seguindo os horários e habilidades de cada funcionário. \\ \hline
            Diretor de Gestão de Pessoas
              & Diretor a qual fica a cargo de gerenciar os recursos humanos da InSoft.
              & Adicionar novos funcionários no sistema atendendo suas habilidades e horário;
                Retirar registro dos funcionários que foram desligados da empresa.
              & Envolvido com a área de seleção e cadastro de cada novo funcionário da InSoft, sendo também responsável pela área de desligamento e remoção de cadastros. \\ \hline
        \end{tabular}
        \caption{Resumo dos usuários}
      \end{table}

    \subsection{Ambiente do Usuário}

      A inSoft possui XX funcionários, sendo que em cada projeto de desenvolvimento pretende-se alocar 4 funcionários. Atualmente não há dados sobre o tempo de execução de tarefas da área de desenvolvimento devido ao pouco tempo que a empresa encontra-se em operação, e por não haver concluído nenhum projeto de desenvolvimento até o presente momento.

      Embora a inSoft utilize atualmente o Trello no gerenciamento de suas atividades, o mesmo não consegue cumprir com as necessidades da empresa, devendo ser substituido pelo SGi. Também são utilizados documentos eletrônicos para o gerenciamento de alguns setores da empresa. O setor de Recursos Humanos, por exemplo, utiliza planilhas eletrônicas para o gerenciamento das informações sobre os membros da inSoft, e por conta disto é necessários que se desenvolva uma funcionalidade que permita a importação desses dados gravados em documentos eletrônicos.

      Por não possuir ambiente físico estável, ou equipamentos eletrônicos próprios, a inSoft prefere trabalhar com sistemas de \emph{software} \emph{online}.

    \subsection{Resumo das Principais Necessidades dos Envolvidos ou dos Usuários}

      As atuais  ferramentas utilizadas no gerenciamento da equipe de desenvolvimento provocam os seguintes problemas:

      \begin{itemize}
        \item Falta de visibilidade das habilidades dos funcionários
        \item Falta de visibilidade da disponibilidade de horário dos funcionários
        \item Falta de visibilidade das tarefas alocadas
        \item Falta de visibilidade dos responsáveis por cada tarefa
        \item Falta de visibilidade quanto a documentação produzida pela equipe de desenvolvimento
        \item Dificuldade de gerir os documentos produzidos pela equipe de desenvolvimento
        \item Dificuldade de gerir os responsáveis por cada atividade
        \item Dificuldade de gerir a alocação de tarefas
      \end{itemize}

      Todos os problemas supracitados são causados pela dificuldade em convergir todas as informações registradas pela inSoft tanto em documentos eletrônicos, como em ferramentas de gerenciamento de projeto que utilizam.

      Embora a inSoft tente estabelecer uma metodologia de organização sistemática, a falta de centralização de todas as informações importantes para o gerenciamento dos projetos de desenvolvimento, bem como da equipe de desenvolvimento, tem frustado suas tentativas de contornar esses problemas.

      A fim de resolver esses problemas, a inSoft espera que o SGi possibilite, de forma centralizada, o controle sobre as informações dos funcionários, a disponibilidades dos mesmos; a organização dos projetos de desenvolvimento e suas atividades; e a possibilidade de conseguir alocar facilmente funcionários para um determinado projeto e atividade sem causar conflitos de disponibilidade de horários.

    \subsection{Alternativas e Concorrência}
       \label{subsec:Concorrentes}
      \begin{table}[H]
        \centering
        \begin{tabular}{|p{0.1\textwidth}|p{0.26\textwidth}|p{0.26\textwidth}|p{0.26\textwidth}|}
          \hline
          \rowcolor[HTML]{C0C0C0}
          \multicolumn{1}{c}{Produto} & \multicolumn{1}{|c|}{Afeta qual problema ?} & \multicolumn{1}{|c|}{Pontos fortes} & \multicolumn{1}{|c|}{Pontos Fracos} \\ \hline
            Nex
              & Falta de visibilidade de informações de cada membro da equipe.
              & Sistema de cadastro simples;
                Pode ser usado na parte de controle de finanças da InSoft.
              & Muito genérico e não atende todas as necessidades esperadas para resolver o problema, como mostrar horários disponíveis e o quadro de habilidades de cada usuário. \\ \hline
            Trello
              & Membros não têm visibilidade dos processos da empresa.
              & Mostra todos as atividades para a equipe de desenvolvimento sobre o como está o andamento do projeto de tal Software.
              & Não integra tudo que o sistema necessita em relação ao desenvolvimento de software, mas sim um Kanban simples com o fluxo de atividades;
                Necessita cadastro e não possui modo de exportar, fazendo assim a empresa refém do serviço que pode deixar de ser prestado a qualquer momento. \\ \hline
            Arivo
              & Falta de visibilidade de informações de cada membro da equipe;
                Membros não têm visibilidade dos processos da empresa.
              & Consegue unir a parte de cadastro de clientes, junto com uma área de modelo de processo, assim podendo resolver dois problemas em um.
              & Pago;
                Criado originalmente para CRM (Customer Relationship Management), ferramenta não atrelada com a função base a qual seria inserida na InSoft, podendo assim causar certas inconsistências. \\ \hline
        \end{tabular}
        \caption{Alternativas e concorrência}
      \end{table}

  \section{Visão Geral do Produto}

    \subsection{Perspectiva do Produto}

      Dado que esse produto tem como meta resolver os problemas de falta de controle dos processos, referentes a área de desenvolvimento dentro da empresa inSoft, fornecendo um meio de facilitar e organizar a visibilidade desses processos. Funciona como uma ferramenta de gestão para manter e registrar as informações referentes a disponibilidade de horários e habilidades especificas de cada membro da empresa e alocá-los às suas devidas tarefas e atividades, nas quais são mais justas de acordo com suas informações registradas na ferramenta.

      Como já foi citado, essa ferramenta provê uma melhor visibilidade do processo afim de servir como meio de auxílio da parte de gestão da empresa, que além de se evitar possíveis falhas no controle do processo, aumentaria a eficiência e consequentemente a produtividade dentro dos processos da empresa.

    \subsection{Suposições e Dependências}
  \section{Recursos do Produto}
  \section{Outros Requisitos do Produto}
